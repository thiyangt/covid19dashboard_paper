% interactcadsample.tex
% v1.03 - April 2017

\documentclass[]{interact}

\usepackage{epstopdf}% To incorporate .eps illustrations using PDFLaTeX, etc.
\usepackage{subfigure}% Support for small, `sub' figures and tables
%\usepackage[nolists,tablesfirst]{endfloat}% To `separate' figures and tables from text if required

\usepackage{natbib}% Citation support using natbib.sty
\bibpunct[, ]{(}{)}{;}{a}{}{,}% Citation support using natbib.sty
\renewcommand\bibfont{\fontsize{10}{12}\selectfont}% Bibliography support using natbib.sty

\theoremstyle{plain}% Theorem-like structures provided by amsthm.sty
\newtheorem{theorem}{Theorem}[section]
\newtheorem{lemma}[theorem]{Lemma}
\newtheorem{corollary}[theorem]{Corollary}
\newtheorem{proposition}[theorem]{Proposition}

\theoremstyle{definition}
\newtheorem{definition}[theorem]{Definition}
\newtheorem{example}[theorem]{Example}

\theoremstyle{remark}
\newtheorem{remark}{Remark}
\newtheorem{notation}{Notation}

% see https://stackoverflow.com/a/47122900

% Pandoc citation processing

\usepackage{hyperref}
\usepackage[utf8]{inputenc}
\def\tightlist{}


\begin{document}

\articletype{}

\title{Interactive dashboard to monitor the COVID-19 outbreak}


\author{\name{Thiyanga S. Talagala$^{a}$, Randi Shashikala$^{a}$}
\affil{$^{a}$Department of Statistics, University of Sri Jayeardenepura}
}

\thanks{CONTACT Thiyanga S.
Talagala. Email: \href{mailto:ttalagala@sjp.ac.lk}{\nolinkurl{ttalagala@sjp.ac.lk}}, Randi
Shashikala. Email: }

\maketitle

\begin{abstract}
As of September 20th, 2021, 221 countries and territories are infected
by the COVID-19 worldwide pandemic. Dashboards are the most often used
visualization method for visualizing COVID-19 data and informing the
public. The main objective of this study is to identify different
features, visualization methods \& improvements that should be occurred
by exploring the existing dashboards and to develop a dashbored to
visualize COVID-19 outbreak in Sri Lanka. We explored 15 different
dashboards. The most commonly used visualization methods in dashboard
development are bar charts, line charts, and interactive maps.
Dashboards that fit on a single screen are preferable than others.
\end{abstract}

\begin{keywords}
COVID-19, Dashboard, Visualization
\end{keywords}

\hypertarget{introduction}{%
\section{Introduction}\label{introduction}}

The COVID-19 pandemic is a global coronavirus illness outbreak that
began in 2019 and is being caused by the Severe Acute Respiratory
Syndrome Coronavirus 2 (SARS-CoV-2) virus. In December 2019, the first
COVID-19-infected patient was discovered in Wuhan, China. According to
Worldmeter data, there were 229835231 confirmed cases, 206515718
recovered cases, 18605877 active cases, and 4713636 deaths worldwide as
of September 20, 2021. Currently, 99.5 percent of active patients are in
a mild state, while 0.5 percent are in a critical state.

Dashboards are one of the greatest visual interpretation methods for
tracking the COVID-19 pandemic's spread and communication. There are a
plethora of COVID-19 visualization dashboards that have been designed to
visualize the pandemic's global and local status. Different software can
be used to generate dashboards. We explored 15 dashboards designed to
visualize COVID-19 data in the global and country levels. First,
dashboards were compared to identify the various features, visualization
approaches, and enhancements that should be implemented. Next, we
developed an interactive dashboard to visualized COVID-19 outbreak in
Sri Lanka.

\hypertarget{literature-review}{%
\section{Literature Review}\label{literature-review}}

\hypertarget{methodology}{%
\section{Methodology}\label{methodology}}

\hypertarget{results}{%
\section{Results}\label{results}}

\bibliographystyle{tfcad}
\bibliography{interactcadsample.bib}


\input{"appendix.tex"}


\end{document}
